\documentclass[10pt,a4paper]{article}
\usepackage[utf8]{inputenc}
\usepackage[german]{babel}
\usepackage{amsmath}
\usepackage{amsfonts}
\usepackage{amssymb}

\title{Versuchsprotokoll}

\begin{document}
\maketitle
\section*{Einleitung}
Accessmail ist ein einfach zu bedienender Email-Client. In folgenden Tests 
sollen die Funktionalitäten untersucht werden. Insbesondere 
soll auf die Einfachheit der Abläufe in der GUI geachtet werden.
\section*{Versuche}
\subsection*{Versuch I}
Erstellen eines Email Accounts
\begin{itemize}
	\item Besuchen Sie folgende Homepage "'www.hotmail.de"'.
	\item Erstellen Sie sich einen Email Account.
	\item Bei Problemen stellen wir einen Email Account für spätere Versuche.
\end{itemize}
\subsection*{Versuch II}
Emails schicken und empfangen und beantworten.
\begin{itemize}
	\item Öffnen Sie das Programm.
	\item Geben Sie die Daten des vorher erstellten E-mail Accounts ein.
	\item Gehen Sie auf den Reiter "'Write"' und verfassen eine E-mail an unseren Testaccount "'accessTest@hotmail.de"' selbst. 
	\item Schicken Sie die E-mail ab.
	\item Wechseln Sie auf den Reiter "'Overview"' und warten sie auf die Antwort des Admin Accounts.
	\item Öffnen Sie die Mail und lesen Sie sie.
	\item Beantworten Sie die Mail über die Schaltfläche "'Reply"'.	
\end{itemize}
\subsection*{Versuch III}
Kontakte hinzufügen und löschen.
\begin{itemize}
	\item Öffnen sie das Programm (falls noch nicht geschehen).
	\item Gehen sie auf den Reiter "'Kontakte"'
	\item Fügen sie einen neuen Kontakt für "'accessTest@hotmail.de"' hinzu.
	\item Fügen sie einen zweiten Kontakt ihrer Wahl hinzu.
	\item Löschen sie den Kontakt für "'accessTest@hotmail.de"'. 	
\end{itemize}
\section*{Feedback}
Geben sie konstruktives Feedback anhand ihrer Gedanken zu den Versuchen! 
\end{document}