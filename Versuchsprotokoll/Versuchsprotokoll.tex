\documentclass[10pt,a4paper]{article}
\usepackage[utf8]{inputenc}
\usepackage[german]{babel}
\usepackage{amsmath}
\usepackage{amsfonts}
\usepackage{amssymb}

\title{Versuchsprotokoll}

\begin{document}
\maketitle
\section*{Einleitung}
Accessmail ist ein einfach zu bedienender Email-Client. In folgenden Tests 
sollen die Funktionalitäten untersucht werden. Insbesondere 
soll auf die Einfachheit der Abläufe in der GUI geachtet werden.
\section*{Versuche}
% Nur email account erstellen
% Test wie weit das Medium Email verbreitet ist
% Falls der Provider von unseren angebotenen abweicht, einfach noch einen erstellen lassen
% eigene Emails können evtl. auch verwendet werden
\subsection*{Versuch I}
Erstellen eines Email Accounts
\begin{itemize}
	\item Erstellen Sie sich einen Email Account bei einem der folgenden Anbieter: gmail.com oder web.de
	\item Sollten sie schon einen Account bei einem der Anbieter besitzen, koennen sie diesen gerne fuer die folgenden Versuche benutzen.
	\item Bei Problemen stellen wir einen Email Account für spätere Versuche.
\end{itemize}
\subsection*{Versuch II}
Kontakte erstellen und loeschen.
\begin{itemize}
	\item Öffnen Sie das Programm.
	\item Geben Sie die Daten des vorher erstellten E-mail Accounts ein.
	\item Erlautern Sie kurz welches Icon sie womit in Verbindung bringen.
	\item Öffnen Sie das Addressbuch und fügen sie die Addresse ihres Nachbarn hinzu.
	\item Fügen Sie außerdem eine willkürliche Addresse hinzu.
	\item Löschen Sie diese wieder.
	\item Fügen Sie weiter ein paar Addressen hinzu und machen sie sich mit dem Addressbuch vertraut	
\end{itemize}
\subsection*{Versuch III}
Emails verfassen und lesen.
\begin{itemize}
	\item Öffnen Sie das Programm (falls noch nicht geschehen).
	\item Gehen Sie auf den Reiter Addressbuch und verfassen sie eine Email an einen ihrer Kontakte.
	\item Gehen Sie anschließen auf den Reiter "verfassen" und verfassen sie eine Email an sich selber.
	\item Gehen Sie auf den Reiter "Übersicht" und schauen sie sich ihre vorhandenen Mails an.
	\item Lesen Sie eine der Mails und beantworten sie diese.
	\item Löschen Sie eine Mail und drücken sie auf den Button zum aktualisieren ihres Posteinganges. 	
\end{itemize}
\section*{Feedback}
Beantworten Sie bitte die folgenden Fragen:
Wie hat dir das Design der Software gefallen? (Farben, Bilder, Struktur)
Ist dir die Bedienung schwer gefallen? Falls Ja, was genau?
War es schwierig Kontakte anzulegen/zu löschen, Emails zu schreiben/zu löschen?
Welche Funktion würdest du dir zusätzlich wünschen?
Gibt es sonst irgendwelche Anmerkungen?
Vielen dank das Sie sich die Mühe genommen haben unser Programm zu testen!
\end{document}