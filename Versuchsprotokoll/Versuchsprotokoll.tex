\documentclass[10pt,a4paper]{article}
\usepackage[utf8]{inputenc}
\usepackage[german]{babel}
\usepackage{amsmath}
\usepackage{amsfonts}
\usepackage{amssymb}

\title{Versuchsprotokoll}

\begin{document}
\maketitle
\section*{Einleitung}
Irgendwie fällt mir grade auf, wie unsinnig es ist, Versuchsprotokolle zu entwerfen, wenn die Software noch 
längst nicht in ihrer vollendeten Version steht...\\
Funktionalität, die untersucht werden sollte: Kontakte, Emails, Accounts, Anhänge?
\section*{Versuche}
\subsection*{Versuch I}
Emails schicken und empfangen und beantworten.
\begin{itemize}
	\item Öffnen Sie das Programm.
	\item Geben Sie die Daten eines vorhandenen E-mail Accounts ein.
	\item Gehen Sie auf den Reiter "Write" und verfassen eine E-mail an sich selbst. 
	\item Schicken Sie die E-mail ab.
	\item Wechseln Sie auf den Reiter "Overview" und suchen Sie nach Ihrer grade geschickten Mail.
	\item Öffnen Sie die Mail und lesen Sie sie.
	\item Beantworten Sie die Mail über die Schaltfläche "Reply".	
\end{itemize}
\subsection*{Versuch II}
Kontakte hinzufügen, emails schicken und löschen.
\begin{itemize}
	\item Öffnen Sie das Programm.
	\item Geben Sie die Daten eines vorhandenen E-mail Accounts ein.
	\item Gehen Sie auf den Reiter "Write" und verfassen eine E-mail an sich selbst. 
	\item Schicken Sie die E-mail ab.
	\item Wechseln Sie auf den Reiter "Overview" und suchen Sie nach Ihrer grade geschickten Mail.
	\item Öffnen Sie die Mail und lesen Sie sie.
	\item Beantworten Sie die Mail über die Schaltfläche "Reply".	
\end{itemize}
\end{document}