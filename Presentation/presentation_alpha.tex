\documentclass[9pt]{beamer}

\usetheme{TUDo}

% Sprachumgebung
\usepackage[ngerman]{babel}


% Encoding je nach Compiler
\ifluatex
\usepackage[utf8]{luainputenc}
\else
\usepackage[utf8]{inputenc}
\usepackage[T1]{fontenc}
\fi


% Mathematik
\usepackage{amsmath}
\usepackage{amsfonts}
\usepackage{amssymb}
\usepackage{cancel}

%Links
\usepackage[]{hyperref}

%%%%%%%%%%%%%%%%%%%%%%%%%%%%%%%%%%%%%%%%%%%%%%%%%%%%%%%%%%%%%%%%%%%%%%%%%%%%%%%%
%%%%%-------------Hier Titel/Autor/Grafik/Lehrstuhl eintragen--------------%%%%%
%%%%%%%%%%%%%%%%%%%%%%%%%%%%%%%%%%%%%%%%%%%%%%%%%%%%%%%%%%%%%%%%%%%%%%%%%%%%%%%%

%Titel:
\title{Accessmail - an accessible E-Mail Client}
%Autor
\author{Simon Demming}
%Lehrstuhl/Fakultät
\institute[Department]{\par\smallskip\smallskip Department for Rehabilitation\\ Department for Computer Science}


\begin{document}

	\begin{frame}
		\setcounter{framenumber}{0}
	    \titlepage
	\end{frame}
	
	\begin{frame}
	    \frametitle{Einführung}
	    \tableofcontents
	\end{frame}
	
	%Einführung
	\section{Introduction}
		\begin{frame}
			\frametitle{Introduction}
			
			E-Mail as a very important communication platform
			\begin{itemize}
				\item Used for direct communication
				\item Used for registration purposes
				\item Used for many other cool things like explosions
			\end{itemize}
		
		\end{frame}
	
	
	% Verwandte Arbeiten
%	\section{Verwandte Arbeiten}
%		\begin{frame}
%			\frametitle{Verwandte Arbeiten}
%			Dieser Abschnitt wird vermutlich noch rausgeschnitten. Grobe Ausführungen über Regions- und 	
%			Texturbasierte Verfahren
%		\end{frame}
	
	
	%Stroke Width Transform
	\section{Our Approach}
		\subsection{Preparations}
		
			\begin{frame}
				\frametitle{Preparations}
				\begin{itemize}
					\item Kantendetektion
					\item Pfaderzeugung
					\item Wertezuweisung
					\item Komponenten zusammenführen
					\item Textzeilen markieren
				\end{itemize}
			\end{frame}
			
			\begin{frame}
				\frametitle{Canny Kantendetektion}
				%TODO: Quellen http://upload.wikimedia.org/wikipedia/commons/f/f0/Valve_original_%281%29.PNG
		    		
			\end{frame}
			
	
		\subsection{Bestimmung der SWT-Werte}
			\begin{frame}
				\frametitle{Bestimmung der SWT-Werte}
		    		%Bild, welches die Pfadverfolgung von Kante zu Kante anzeigt
		    		
			\end{frame}
			
			\begin{frame}
				\frametitle{Bestimmung der SWT-Werte}
		    		%Bild, welches die Pfadverfolgung von Kante zu Kante anzeigt
		    		
			\end{frame}
			
			\begin{frame}
				\frametitle{Bestimmung der SWT-Werte}
		    		%Bild, welches die Pfadverfolgung von Kante zu Kante anzeigt
		    		
			\end{frame}
			
			\begin{frame}
				\frametitle{Bestimmung der SWT-Werte}
		    		%Bild, welches die Pfadverfolgung von Kante zu Kante anzeigt
		    		
			\end{frame}
			
			\begin{frame}
				\frametitle{Bestimmung der SWT-Werte}
		    		%Bild, welches die Pfadverfolgung von Kante zu Kante anzeigt
		    		
			\end{frame}
			
			\begin{frame}
				\frametitle{Bestimmung der SWT-Werte}
		    		%Bild, welches die Pfadverfolgung von Kante zu Kante anzeigt
		    		
			\end{frame}
			
			\begin{frame}
				\frametitle{Bestimmung der SWT-Werte}
		    		%Bild, welches die Pfadverfolgung von Kante zu Kante anzeigt
		    		
			\end{frame}
	
		\subsection{Connected Components}
			\begin{frame}
				\frametitle{Connected Components}
				
		    	\end{frame}
			
			\begin{frame}
				\frametitle{Connected Components}
				
		    	\end{frame}
		    		
		    	\begin{frame}
		    		\frametitle{Connected Components}
		    		
				
				%Bild zum Algorithmus, vllt. das Bild mit dem Männchen von Wikipedia? http://upload.wikimedia.org/wikipedia/commons/thumb/b/b9/Two-pass_connected_component_labeling.svg/220px-Two-pass_connected_component_labeling.svg.png
				%s.  http://www.ub.tu-dortmund.de/katalog/titel/1304452
			\end{frame}
	
		\subsection{Komponenten erkennen}
			\begin{frame}
				\frametitle{Buchstaben zu Textzeilen gruppieren}
				
			\end{frame}
			
			% http://habrastorage.org/storage2/c86/5b3/67c/c865b367ce6a23bc0eda9423f9b29fc1.jpg
			
			\begin{frame}
				\frametitle{Buchstaben zu Textzeilen gruppieren}
			
			\end{frame}
	
	\section{Evaluation}
		\subsection{Datensätze}
		\begin{frame}
			\frametitle{ICDAR}
		\end{frame}
	
		\subsection{Gütemaße}
		\begin{frame}
			\frametitle{Gütemaße}
			$
		    \begin{array}{lcl}
				Precision & = & \frac{Ground\:Truth \:\cap\: Gefundene\:Markierungen}{Gefundene\:Markierungen} \\
				\\
				Recall & = & \frac{Ground\:Truth \:\cap\: Gefundene\:Markierungen}{Ground\:Truth} \\
				\\
				f-measure & = & \frac{1}{\frac{\alpha}{Precision}+\frac{1-\alpha}{Recall}}
			\end{array}
			$
		\end{frame}
	
	\section{Zusammenfassung}
		\begin{frame}
			\frametitle{Zusammenfassung}
			%Diese Zusammenfassung ist nicht gut.
			\begin{itemize}
				\item Strichbreite als Indikator für Text
				\item Komponenten werden aus SWT-Werten gebildet
				\item Textzeilen und Wörter werden im Ursprungsbild indiziert
			\end{itemize}
			
		\end{frame}

\end{document}
